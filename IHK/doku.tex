\documentclass[12pt]{article}

\usepackage{ragged2e}
\justifying

\usepackage{helvet}
\renewcommand{\familydefault}{\sfdefault}

\usepackage{microtype}
\usepackage{float}
\usepackage{hyperref}
\renewcommand{\tablename}{Tabelle}

\title{Beschte Onboarding}
\date{\today}
\author{Julian Thiele}


\begin{document}
\sloppy

\maketitle
\newpage

%---------------------------------------------------------------
%
%                   INTRODUCTION
%
%---------------------------------------------------------------

\section{Einleitung}

\subsection{Ausbildungsbetrieb}
Die BREDEX GmbH wurde im Jahr 1987 in Braunschweig gegründet. Den Schwerpunkt 
bildet die individuelle Softwareentwicklung, die BREDEX führt jedoch auch 
Beratungen zu Datenschutz, Datensicherheit, Qualitätssicherung von Software 
und Schulungen durch. Die BREDEX GmbH besitzt, mit seiner Tochterfirma BREDEX 
HUNGARY KFT. zusammen, ca. 200 Mitarbeiter. 

\subsection{Projektumfeld}
Das Projekt wird unter der Aufsicht von Herr Christian Rucinski in den 
Geschäftsräumen der BREDEX GmbH entwickelt. Bei der Anwendung handelt es sich 
um ein BREDEX internes Tool. Sie soll für den Onboardingprozess neuer Mitarbeiter 
eingesetzt werden und den sogenannten BREDEX Buddy, einen Ansprechpartner 
neuer Mitarbeiter, bei seiner Arbeit unterstützen. 

Das Frontend wird mit dem Framework Vue, bei welchem die Scriptsprache TypeScript 
Anwendung findet, geschrieben. Als Backend wird Firebase als “Backend As a Service” 
genutzt, in welche die NoSQL-Datenbank Firestore integriert ist.

\subsection{Projektabgrenzung}
Durch die Fachinformatikerausbildungsverordung wird das Projekt zeitlich auf 80 
Stunden begrenzt.  

Die Anwendung ist ein alleinstehendes Tool und wird daher in kein Projekt 
eingegliedert. Daher gibt es keine weiteren Beschränkungen, auf die geachtet 
werden muss.  

\subsection{Abweichungen vom Projektantrag}
Es gibt keine Abweichungen vom Projektantrag.


%---------------------------------------------------------------
%
%                   Projektplanung
%
%---------------------------------------------------------------

\section{Projektplanung}

\subsection{Identifizierung der Arbeitspakete}
Zu Beginn findet ein Anforderungsgespräch statt, bei dem die Projektziele 
definiert werden und es keine Differenzen im Verständnis dieser gibt und es 
definiert ist, welche Anforderungen erfüllt werden müssen. 

Im Anschluss kommt die Analysephase, bei der Soll-Zustand niedergeschrieben wird.  

Das dritte Paket bildet die Designphase, hierbei werden Benutzeroberfläche und 
Datenmodell geplant.  

Daraufhin folg die Entwicklung der Anwendung, bei der die Ergebnisse aus den 
vorherigen Arbeitspaketen umgesetzt werden. Um mögliche Fehler während der 
Implementierung frühzeitig zu erkennen und beheben zu können, finden Tests statt. 

Im letzten Arbeitsschritt wird die Dokumentation des Projektes erstellt. 

\subsection{Zeitplan}
Das Projekt wird im Zeitraum vom 22.09.2023 bis zum 22.11.2023 bearbeitet. 
Die im vorherigen Abschnitt erwähnten Arbeitspakete wurden in folgenden Zeitplan 
(\hyperref[table:timeplan]{Tabelle \ref*{table:timeplan}}) eingeteilt. 

\begin{table}[H]
    \centering
    \begin{tabular}{|l | l|}
        \hline
        Arbeitspaket & Dauer (in h) \\
        \hline
        Anforderungsgespräch & 3 \\
        Analysephase & 6 \\
        Planungsphase & 6 \\
        Designphase & 6 \\
        Umsetzung & 40 \\
        Test und Abnahme & 7 \\
        Dokumentation & 12 \\
        &\\
        Gesamt & 80 \\
        \hline
    \end{tabular}
    \caption{Zeitplan}
    \label{table:timeplan}
\end{table}

\subsection{Ressourcenplanung}
Um die kosten so gering wie möglich zu halten, wurde nach Möglichkeit nur
kostenlose oder bereits vorhandene Software, sowie Hardware genutzt.
Als Arbeitsgerät wurde von der BREDEX GmbH ein Laptop mit Windows 10 und der
Entwicklungsumgebung VisualStudio Code bereitgestellt.
Des Weiteren wurde Firabase mit Firestore als Datenbank und Git als 
Versionsverwaltung eingesetzt.

\subsection{Kostenplanung}
Für den Prüfungsteilnehmer wird ein Stundenlohn von 9 EUR angesetzt.
Ihm stehen für dei Bearbeitung 80 Stunden zur Verfügung.
Der Projektbetreuer führt das Anforderungsgespräch durch und steht 
dazu noch für weitere Fragen während der Bearbeitung zur Verfügung.
Daher werden für ihn 6 Stunden angesetzt, bei einem geschätzten 
Stundenlohn von 45 EUR
Da nur bereits vorhandene oder kostenlose Ressourcen genutzt werden,
ergeben sich keine weiteren Lizenz- oder Nutzungskosten.

Als Gesamtkosten berechnen sich hiermit 990 EUR 
(\hyperref[table:costs]{Tabelle \ref*{table:costs}}).

\begin{table}[H]
    \centering
    \begin{tabular}{|l|l|l|l|}
        \hline
        Akteur & Stundensatz (in EUR) & Dauer (in h) & Kosten (in EUR) \\
        \hline
        Prüfungsteilnehmer & 9 & 80 & 720 \\
        Projektbetreuer & 45 & 6 & 270 \\
        \textbf{Gesamtkosten} &&& \textbf{990} \\
        \hline
        
    \end{tabular}
    \caption{Kostenplan}
    \label{table:costs}
\end{table}


\subsection{Amortisaionsdauer}

Keine Ahnung, wie die genau geht :D



%---------------------------------------------------------------
%
%                   Analysephase
%
%---------------------------------------------------------------

\section{Analysephase}

\subsection{Anforderungsgespräch}
Als Erstes wurde ein Anfürderungsgespräch mit dem Kunden durchgeführt.
In dem Gespräch wird der IST-Zustand ermittelt und das SOLL-Konzept definiert.



%---------------------------------------------------------------
%
%                   Design
%
%---------------------------------------------------------------

\section{Designphase}



%---------------------------------------------------------------
%
%                   Implementierung
%
%---------------------------------------------------------------

\section{Implementierung}



%---------------------------------------------------------------
%
%                   Testphase
%
%---------------------------------------------------------------

\section{Testphase}



%---------------------------------------------------------------
%
%                   Fazit
%
%---------------------------------------------------------------

\section{Fazit}



\end{document}