\documentclass[12pt]{article}

\usepackage{ragged2e}
\justifying

\usepackage{helvet}
\renewcommand{\familydefault}{\sfdefault}

\usepackage[ngerman]{babel}

\usepackage{pifont}

\usepackage{microtype}
\usepackage{float}
\usepackage{hyperref}

\title{Beschte Onboarding}
\date{\today}
\author{Julian Thiele}


\begin{document}
\sloppy

\maketitle
\newpage

\tableofcontents
\newpage

%---------------------------------------------------------------
%
%                   INTRODUCTION
%
%---------------------------------------------------------------

\section{Einleitung}

\subsection{Ausbildungsbetrieb}
Die BREDEX GmbH wurde im Jahr 1987 in Braunschweig gegründet. Den Schwerpunkt 
bildet die individuelle Softwareentwicklung, die BREDEX führt jedoch auch 
Beratungen zu Datenschutz, Datensicherheit, Qualitätssicherung von Software 
und Schulungen durch. Die BREDEX GmbH besitzt, mit seiner Tochterfirma BREDEX 
HUNGARY KFT. zusammen, ca. 200 Mitarbeiter. 

\subsection{Projektumfeld}
Das Projekt wird unter der Aufsicht von Herr Christian Rucinski in den 
Geschäftsräumen der BREDEX GmbH entwickelt. Bei der Anwendung handelt es sich 
um ein BREDEX internes Tool. Sie soll für den Onboardingprozess neuer Mitarbeiter 
eingesetzt werden und den sogenannten BREDEX Buddy, einen Ansprechpartner 
neuer Mitarbeiter, bei seiner Arbeit unterstützen. 

Das Frontend wird mit dem Framework Vue, bei welchem die Scriptsprache TypeScript 
Anwendung findet, geschrieben. Als Backend wird Firebase als “Backend As a Service” 
genutzt, in welche die NoSQL-Datenbank Firestore integriert ist.

\subsection{Projektabgrenzung}
Durch die Fachinformatikerausbildungsverordung wird das Projekt zeitlich auf 80 
Stunden begrenzt.  

Die Anwendung ist ein alleinstehendes Tool und wird daher in kein Projekt 
eingegliedert. Daher gibt es keine weiteren Beschränkungen, auf die geachtet 
werden muss.  

\subsection{Abweichungen vom Projektantrag}
Es gibt keine Abweichungen vom Projektantrag.


%---------------------------------------------------------------
%
%                   Projektplanung
%
%---------------------------------------------------------------

\section{Projektplanung}

\subsection{Identifizierung der Arbeitspakete}
Zu Beginn findet ein Anforderungsgespräch statt, bei dem die Projektziele 
definiert werden und es keine Differenzen im Verständnis dieser gibt und es 
definiert ist, welche Anforderungen erfüllt werden müssen. 

Im Anschluss kommt die Analysephase, bei der Soll-Zustand niedergeschrieben wird.  

Das dritte Paket bildet die Designphase, hierbei werden Benutzeroberfläche und 
Datenmodell geplant.  

Daraufhin folg die Entwicklung der Anwendung, bei der die Ergebnisse aus den 
vorherigen Arbeitspaketen umgesetzt werden. Um mögliche Fehler während der 
Implementierung frühzeitig zu erkennen und beheben zu können, finden Tests statt. 

Im letzten Arbeitsschritt wird die Dokumentation des Projektes erstellt. 

\subsection{Zeitplan}
Das Projekt wird im Zeitraum vom 22.09.2023 bis zum 22.11.2023 bearbeitet. 
Die im vorherigen Abschnitt erwähnten Arbeitspakete wurden in folgenden Zeitplan 
(\autoref{table:timeplan}) eingeteilt. 

\begin{table}[H]
    \centering
    \begin{tabular}{|l | l|}
        \hline
        Arbeitspaket & Dauer (in h) \\
        \hline
        Anforderungsgespräch & 3 \\
        Analysephase & 6 \\
        Planungsphase & 6 \\
        Designphase & 6 \\
        Umsetzung & 40 \\
        Test und Abnahme & 7 \\
        Dokumentation & 12 \\
        &\\
        Gesamt & 80 \\
        \hline
    \end{tabular}
    \caption{Zeitplan}
    \label{table:timeplan}
\end{table}

\subsection{Ressourcenplanung}
Um die kosten so gering wie möglich zu halten, wurde nach Möglichkeit nur
kostenlose oder bereits vorhandene Software, sowie Hardware genutzt.
Als Arbeitsgerät wurde von der BREDEX GmbH ein Laptop mit Windows 10 und der
Entwicklungsumgebung VisualStudio Code bereitgestellt.
Des Weiteren wurde Firabase mit Firestore als Datenbank und Git als 
Versionsverwaltung eingesetzt.

\subsection{Kostenplanung}
Für den Prüfungsteilnehmer wird ein Stundenlohn von 9 EUR angesetzt.
Ihm stehen für dei Bearbeitung 80 Stunden zur Verfügung.
Der Projektbetreuer führt das Anforderungsgespräch durch und steht 
dazu noch für weitere Fragen während der Bearbeitung zur Verfügung.
Daher werden für ihn 6 Stunden angesetzt, bei einem geschätzten 
Stundenlohn von 45 EUR
Da nur bereits vorhandene oder kostenlose Ressourcen genutzt werden,
ergeben sich keine weiteren Lizenz- oder Nutzungskosten.

Als Gesamtkosten berechnen sich hiermit 990 EUR (\autoref{table:costs}).

\begin{table}[H]
    \centering
    \begin{tabular}{|l|l|l|l|}
        \hline
        Akteur & Stundensatz (in EUR) & Dauer (in h) & Kosten (in EUR) \\
        \hline
        Prüfungsteilnehmer & 9 & 80 & 720 \\
        Projektbetreuer & 45 & 6 & 270 \\
        \hline
        \textbf{Gesamtkosten} &&& \textbf{990} \\
        \hline
        
    \end{tabular}
    \caption{Kostenplan}
    \label{table:costs}
\end{table}


\subsection{Nutzwertanalyse}

Kommt noch :D



%---------------------------------------------------------------
%
%                   Analysephase
%
%---------------------------------------------------------------

\section{Analysephase}

\subsection{Anforderungsgespräch}
Als Erstes wurde ein Anfürderungsgespräch mit dem Kunden durchgeführt.
In dem Gespräch wird der IST-Zustand ermittelt und das SOLL-Konzept definiert.
Außerdem wurden die vom Kunden gewünschten Features der Amwendung übermittelt.
Die wichtigsten Features der Anwendung sind:

\begin{itemize}
    \item Achievements, die abgehakt werden können
    \item Rangliste, um sich mit Mitarbeiter*innen vergleichen zu können
    \item Als Administrator Achievementliste bearbeiten können 
\end{itemize}

Dei Punkte wurden in dem Gespräch noch weiter konkretisiert, um die Anwendung
im Anschluss möglichst nach den Wünschen des Kunden implementieren zu können.


\subsection{IST-Zustand Analyse}

Damit neue Mitarbeiter*innen der BREDEX GmbH besser in das Unternehmen
finden, werden ihnen ein Ansprechpartner, den sogenannten BREDEX-Buddy
an die Seite gestellt. Dieser soll dem*der Mitarbeiter*in das Unternehmen
näherbringen. Allerdings ist nicht genau klar, was die Aufgaben des Buddys
alles beinhalten. Es gibt unterschiedlichen Orten findet man Listen mit
verschiedenen Inhalten. 
Dadurch führen die Buddys ihre Aufgabe unterschiedlich aus.


\subsection{SOLL-Konzept}

Um in Zukunft dafür zu sorgen, dass alle Buddys ihre Aufgaben auf die gleiche Art
und Weise zu erledigen soll es eine Webanwendung geben, bei der es eine Checkliste gibt,
die neue Mitarbeiter*innen mit ihren Buddys abarbeiten können.

Zudem soll es eine Rangliste geben, in der sich Mitarbeiter*innen gegnseitig
sehen können, die zur selben Zeit angefangen haben. Diese Gamification soll
die Motivation erhöhen, die Achievements zu erledigen, wodurch die neuen
Mitarbeiter*innen sich mehr in das Unternehmen eingliedern.


%---------------------------------------------------------------
%
%                   Design
%
%---------------------------------------------------------------

\section{Designphase}

\subsection{Benutzeroberfläche}

Bei dem Design der Benutzeroberfläche (im Fologenden GUI genannt) wird
besonders darauf geachtet, dass alle Daten möglichst übersichtlich
dargestellt werden. Dazu werden für jede der großen Anforderungen
jeweils einzelne Unterseiten (im folgendnen View genannt) unterteilt. 

Hierbei gibt es eine View, um alle Achievements einehen zu können
und diese abhaken zu können. Achievements werden zu verschiedenen
Zeitpunkten freigeschaltet. Manche sind direkt verfügbar, machnche 
nach einer Woche, einem Monat oder erst nach einem halben Jahr. 
Entsprechend dieser Zeitstmpel sind die Achievements in Akkordeons
gruppiert. Dies führt dazu, dass Achievements übersichtlich einsehbar
sind.

Eine weitere View gibt es für die Ansicht der Rangliste. Hier wird
der derzeit eingeloggten Person eine Rangliste mit den Personen die
zu einer ähnlichen Zeit angefangen habe angezeigt, mit Name, Email 
und Punktzahl.

Eine letzte View gibt es, um die Achievements verwalten zu können.
Die View ist ähnlich aufgebaut, wie die Achievements-View. Die Achievements
werden in Akkordeons gruppiert, jedoch befindet sich neben jedem Titel
der Akkordeons ein weiterer Button, um neue Achievments für den Zeitpunkt
hinzufügen zu können. Außerdem ist der Button um Achievements abzuhaken
asugetauscht durch jeweils einen Button um das Achievement löschen oder 
bearbeiten zu können.

Beim Hinzufügen oder Bearbeiten eines Achievements öffnet sich ein
Modaldialog, bei dem sich die Daten bearbeiten lassen und dann der
Vorgang abgeschlossen werden kann. Hierbei werden fehlende Eingaben durch
eine rote Umrandung des Eingabefeldes vermittelt.


\subsection{Datenbank}

Bei der Datenbank handelt es sich um die NoSQL Datenbank Firestore
von Firebase. Die Daten werden in JSON-Format gespeichert. Hierbei wird
von Sammlungen und Attributen gesprochen. Sammlungen JSON-Objekte, mit einer
ID und einem Datensatz. Die Daten können entweder aus weiteren Sammlungen
oder Attributen wie Intgern, Arrays oder ähnlichem bestehen.

Um die Achievements und die Daten der Nutzer der Anwendung speichern zu können
wurden in der Datenbank jeweils Sammlungen angelegt.


%---------------------------------------------------------------
%
%                   Implementierung
%
%---------------------------------------------------------------

\section{Implementierungsphase}

\subsection{Frontend}

Im Frontend wird das Webframework Vue eingesetzt. Dieses baut auf den
Standard Websprachen HTML, JavaScript und CSS auf. Vue nutzt
komponentenbasierte Programmierung, indem es HTML um die Möglichkeit
erweitert, Templates zu erzeugen, welche dann entsprechend dem Status des
JavaScript Datei die Benutzeroberfläche erzeugen.
In dieser Anwendung wird Vue-Routing genutzt um eine Single-Page Anwendung zu
erstellen. Dafür wurde sich entschieden, da Single-Page Anwendungen eine 

Standardmäßig wird in Vue die Scriptsprache JavaScript genutzt. Das Framework
kann auch mit der Sprache TypeScript genutzt werden, wofür sich in diesem
Projekt entschieden wurde, um die Vorteile von Typisierung nutzen zu können.
Dadurch können gewisse Fehler bereits zur Kompilierzeit vom Compiler entdeckt werden,
wodurch die Anwendung weniger Fehleranfällig wird.


Vue stellt seit Version 3.0 auch eine neue API zur Verfügung: die Composition API
Die Vorteile der in Vue Standardmäßig genutzten Options API, ist die Möglichkeit
den Code variabler und besser zu strukturieren, was die lesbarkeit und Qualitäts des
Codes verbessert. Außerdem erleichtert die Composition API die Wiederverwendbarkeit
von Code in verschiedenen Komponenten, was eine große Auswirkung auf Codequalität und
Fehleranfällig hat.

Für Styling wird das CSS-Framework Bootstrap genutzt, um einfacher ein modernes
Design für die Anwendung erstellen zu können und sich auf die Porgrammierung des
Projektes kümmern zu können.

View organisiert die Website über mehrere Single-Page-Komponenten. Hierbei handelt
es sich um eine Dateistruktur, bei der jede Datei seine eigene Komponente beinhält.
Die Wurzelkomponente einer jeden Vue-Amwemdung stellt die App-Component dar. In dieser
wird der <vue-router> Tag genutzt. Dadurch lässt sich ein Teil des Inhaltes der App-Komponente
dynamisch mit anderen Komponenten austauschen.

Als erstes wurde die Achievement-View erstellt, da es sich bei der View um die
Hauptanforderung der Website handelt. Hier werden alle Achievements angezeigt, die
der Benutzer erledigen kann und ob diese bereits erledigt wurden.
Im oberen Teil der Website gibt es eine kurze Einführung und eine Fortschrittsleiste
zeigt an, wie viel der Nutzer bereits geschafft hat.
Jede Vue-Komponente hat
hat seinen eigenen Lifcycle. Hierzu gehören beispielsweise das Erstellen, Mounten,
Aktualisieren oder Unmounten der Komponente. Der OnMounted-Lifecycle Hook wird hier genutzt,
um kurz vor dem Anzweigen der Komponente die Daten aus der Datenbank zu laden.
Diese Vorgehensweise wird in anderen Komponenten ebenfalls genutzt, um Daten zu laden. 

Als nächstes wird die Login-View erstellt. Da in der BREDEX GmbH Microsoft Accounts
ganutzt werden, kann sich der Nutzer hier mit diesem Anmelden. Außerdem werden neue Nutzer
erkannt, indem UUIDs verglichen werden und für jeweilige Nutzer neue Einträge in der
Datenbank erstellt.

Im Anschluss wurde die Rangliste erstellt. Beim Laden der Nutzer aus der Datenbank können
zusätzliche Bedingungen angegeben werden. Dadurch werden nur benötigte Daten aus der
Datenbank geladen, um weniger Speicherplatz im lokalen Speicher des Clients speichern zu benötigen.

Die letzte View ist zum Verwalten der Achievements. Auf diese View sollen nur Personen
mit bestimmten Gruppen Zugriff haben, wie beispielsweise die Personalabteilung. Dieser Zugriff
kann mit dem Vue-Router implementiert werden, welcher eine Möglichkeit hat, eine Bedingung
an die Weiterleitung an die Addresse zu binden. In dieser Bdeingung wird die Rolle des
Nutzers abgefragt. Da diese bei der Anmeldung über Firebase nicht mit dem Access-Token
übergeben wird, muss sich separat mit dem Azure Portal verbunden werden, um diese abzufragen.

Um zuletzt die Mehrsprachigkeit umzusetzen, wird sich das Vue Plugin Vue I18N zunutze gemacht.
In einer separaten Datei wird zu jedem Text der Anwendung im JSON-Format ein Key gespreichert und
die jeweiligen Übersetzungen. In dem HTML-Template wird nun anstelle des Textes der Key eingetragen,
welcher beim Erstellen der Komponente, je nach aktuell eingestellter Sprache, durch die Übersetzung
ersetzt wird.

%ref ?


\subsection{Backend}




%---------------------------------------------------------------
%
%                   Testphase
%
%---------------------------------------------------------------

\section{Testphase}



%---------------------------------------------------------------
%
%                   Fazit
%
%---------------------------------------------------------------

\section{Fazit}



\end{document}