\documentclass[11pt]{article}

\usepackage{helvet}
\renewcommand{\familydefault}{\sfdefault}
\usepackage[onehalfspacing]{setspace}
\usepackage{ragged2e}
\justifying

\usepackage[margin=2.5cm]{geometry}

\usepackage[ngerman]{babel}

\usepackage{xcolor}
\definecolor{bx-green}{HTML}{398772}
\usepackage{graphicx}
\graphicspath{{./../pictures/}}
\usepackage{fancyhdr}
\renewcommand{\headrule}{{\color{bx-green}\rule[2ex]{\dimexpr\textwidth-3.55cm}{.95pt}}}
\usepackage{sectsty}
\sectionfont{\color{bx-green}}  % sets colour of sections
\subsectionfont{\color{bx-green}}


\begin{document}

\tableofcontents
\addtocontents{toc}{\protect\thispagestyle{empty}}
\pagenumbering{gobble}
\newpage
\pagenumbering{arabic}
\pagestyle{fancy}
\fancyhead[L]{Benutzerdokumentation - Julian Thiele}
\fancyhead[R]{
    \vspace*{-.4cm}
    \includegraphics[height=.7cm]{bx_logo.png}
}

\section{Startseite}
Beim Aufrufen des Links beschte-onboarding.bredex.de gelangt der Nutzer auf die Login-Seite der Webanwendung.
Über den Login-Button kann der Nutzer sich mit seinem Micorsoft-Account anmelden. Ist dieser auf dem Gerät bereits mit einem
Microsoft-Account angemeldet, wird die Anmeldung automatisch ausgeführt.

%=============================================
%               Login-Page
%=============================================

\section{Allgemeiner Aufbau}

\subsection{Navigation}
Am oberen Rand der Anwendung befindet sich eine Navigationsleiste. Darüber können die verschiedenen Seiten der Anwengung erreicht werden.
Die verschiedenen Seiten der Anwendung sind:
\begin{itemize}
    \item \textbf{Achievements: } Die Hauptseite der Anwendung. Hier können Achievements erledigt werden.
    \item \textbf{Rangliste: } Auf dieser Seite wird die Rangliste dargestellt, bei der man sich mit anderen Nutzern vergleichen kann.
    \item \textbf{Management: } Hier können Achievement bearbeitet werden. Nur ausgewählte Personen habe Zugriff auf diese Seite.
\end{itemize}
%=============================================
%               Navbar-Left
%=============================================

Auf der rechten Seite der Navigationsleiste kann die Sprache der Anwendung zwischen deutsch und Englisch umgestellt werden. Direkt daneben befindet sich der Button,
um sich aus der Anwendung auszuloggen.  
%=============================================
%               Navbar-Right
%=============================================

\subsection{Hinweise}
Direkt unter der Navigationsleiste jeder Seite befindet sich eine kurze Beschreibung, was der Nutzer auf der jeweiligen Seite tun kann.

%=============================================
%               Hinweis
%=============================================

\section{Achievement Liste}
\subsection{Überblick}
Nach der Anmeldung, wird der Nutzer auf die Achievement-Ansicht weitergeleitet. Die Achievements sind hier in 4 Zeitpunkte unterteilt:
\begin{itemize}
    \item Erster Tag
    \item Erste Woche
    \item Erster Monat
    \item Erstes halbes Jahr
\end{itemize}
%=============================================
%               Sektions
%=============================================
Nach Ablauf des Zeitabschnittes seit Arbeitsbeginn des Nutzers bei der BREDEX GmbH werden die Achievements freigeschaltet. Ein Nutzer, der 
bereits seit einem Jahr bei BREDEX arbeitet hat demnach alle Sektionen freigeschaltet, währen einem Nutzer, der erst seit 2 Wochen bei BREDEX
arbeitet nur die ersten beiden Abschnitte zur Verfügung stehen.

\subsection{Achievements erledigen}
Klappt der Nutzer eine Sektion aus, werden die Achievements angezeigt, die in diesem Zeitabschnitt liegen.
Jedes Achievement beinhälte einen Text, den die Aufgabe des Achievements angibt. Darunter werden die Punkt angezeigt, die das jeweilige Achievement gibt.
Auf dem Button in dem Achievement kann das Achievement als erledigt markiert werden. Über denselben Button können bereits
erledigte Achievement wieder als nicht-erledigt markiert werden.
Erledigte Achievements werden in der Darstellung ausgegraut und der Text auf dem Button ändert sich.

%=============================================
%               Achievement
%=============================================

\subsection{Fortschrittsleiste}
Eine Leiste im oberen Teil der Ansicht zeigt dem Nutzer an, wie viele Achievements er bereits erledigt hat.
Ist die Leiste voll, sind alle, bereits für den Nutzer verfügbaren, Achievements erledigt. 

%=============================================
%               Progress bar
%=============================================

\section{Rangliste}
Über die Navigation gelangt der Nutzer auf die Ansicht der Rangliste. 
Welche Nutzer in der Rangliste angezeigt werden, hängt von dem aktuell angemeldeten Nutzer ab.
Damit alle Nutzer in der Rangliste ungefähr ähnliche Vorraussetzungen haben, werden nur andere Nutzer angezeigt,
die höchstens einen Monat vor oder nach dem aktuell angemeldeten Nutzer angefangen haben.

Die Nutzer werden in der Rangliste nach ihrer Platzierung sortiert. Je mehr Punkte ein Nutzer hat, erhält er eine höhere Platzierung.
Die Punkte, die ein Achievement dem Nutzer einbringt, richtet sich nach der Schwierigkeit des Achievements und kann zwischen 0 und 5 Punken variieren.
Nutzer mit gleicher Punktzahl erhalten in der Rangliste die selbe Platzierung.

%=============================================
%               Rangliste
%=============================================

\section{Management}
\subsection{Navigierung zur Management Seite}
Die Management Seite kann von erlaubten Nutzern erreicht werden. Der Link in der Navigationsleiste kann nur von jeweiligen
Nutzern gesehen werden.

\subsection{Achievements anlegen}
Auf der Management Seite sind die Achievements, ähnlich der Achievements Seite, in ihre Zeitabschnitte unterteilt.
Auf der rechten Seite eines jeden Zeitabschnittes kann auf das "Plus" geklickt werden.

%=============================================
%               Anlegen Button
%=============================================


Im Anschluss dazu öffnet sich ein Modaldialog.
Über diesen Dialog können Achievements angelegt werden.
%=============================================
%               Modal
%=============================================
Bei den ersten beiden Textfeldern muss der Nutzer die Aufgabe des Achievements eintragen.
Das erste der beiden Felder ist für den englischen, das Zweite für den deutschen Text.
Fehlt die Angabe in einem der beiden Textfelder, kann der Dialog nicht bestätigt werden.
Das Feld, dessen Angabe fehlt, wird dem Nutzer durch eine Eingabeaufforderung an dem Textfeld angezeigt.
%=============================================
%               Eingabe Aufforderung
%=============================================

Im Anschluss kann der Zeitpunkt für die Freischaltung des Achievements ausgewählt werden.
Je nach dem, neben welchem Abschnitt der Nutzer auf den Button gedrückt hat, ist dieses Feld bereits
vorausgefüllt, kann aber einfach über ein Dropdown Menü im Nachhinein geändert werden.

%=============================================
%               Dropdown
%=============================================

Als letztes kann noch die Punktzahl des Achievements ausgewählt werden. Diese ist standardmäßig auf 1
gesetzt, kann aber wie der Zeitabschnitt über ein Dropdown geändert werden.

%=============================================
%               Punktzahl Dropdown
%=============================================

\subsection{Achievement bearbeiten}
Jedes Achievement kann über einen Button bearbeitet werden.
Dieser Button befindet sich bei dem Achievement selbst.
%=============================================
%               Edit Button
%=============================================
Beim Drücken den Bearbeiten-Buttons, öffnet sich der selbe Dialog, wie beim erstellen eines neuen
Achievements. Der Dialog ist beim bearbeiten vorausgefüllt mit den Daten des Achievements, das der Nutzer gerade
bearbeitet.

\subsection{Achievement löschen}
Neben dem Bearbeiten-Button existiert für jedes Achievement noch ein Button, um dieses zu löschen.
%=============================================
%               Delete Button
%=============================================
Drückt der Nutzer den Button öffnet sich ein Bestätigungsdialog, damit Achievements nicht aus versehen
gelöscht werden können.
%=============================================
%               Confirm Dialog
%=============================================





\end{document}